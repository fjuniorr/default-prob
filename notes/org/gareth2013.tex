% Created 2017-08-13 Sun 10:46
\documentclass[11pt]{article}
\usepackage[utf8]{inputenc}
\usepackage[T1]{fontenc}
\usepackage{fixltx2e}
\usepackage{graphicx}
\usepackage{longtable}
\usepackage{float}
\usepackage{wrapfig}
\usepackage{rotating}
\usepackage[normalem]{ulem}
\usepackage{amsmath}
\usepackage{textcomp}
\usepackage{marvosym}
\usepackage{wasysym}
\usepackage{amssymb}
\usepackage{hyperref}
\tolerance=1000
\author{Francisco Junior}
\date{\today}
\title{gareth2013}
\hypersetup{
  pdfkeywords={},
  pdfsubject={},
  pdfcreator={Emacs 25.2.1 (Org mode 8.2.10)}}
\begin{document}

\maketitle
\tableofcontents


\section{ways to estimate the unknown function $f$ \footnote{$f$ represents the relationship between the inputs and the output}}
\label{sec-1}

\subsection{parametric methods}
\label{sec-1-1}

Its easier to estimate parameters of an assumed functional form than to estimate an arbitrary function

If our assumption is not consistent with the data, the fit could be bad

\subsubsection{step 1 - assumption about the functional form of $f$}
\label{sec-1-1-1}

\begin{enumerate}
\item linear functional form
\label{sec-1-1-1-1}

\item non-linear functional form
\label{sec-1-1-1-2}
\end{enumerate}

\subsubsection{step 2 - estimation of the parameters}
\label{sec-1-1-2}

\subsection{non-parametric methods}
\label{sec-1-2}

Because we need to estimate the function $f$, we need more data.

Our fit will be good, but now we are in danger of following the noise in the data too closely. To avoid this we need to control for the correct amount of smoothness \footnote{in a two dimensional plane for example, this control how rough and wiggly the line is}

\subsubsection{step 1 - no assumption about the functional form}
\label{sec-1-2-1}

\subsubsection{step 2 - estimation of $f$ directly}
\label{sec-1-2-2}
% Emacs 25.2.1 (Org mode 8.2.10)
\end{document}
