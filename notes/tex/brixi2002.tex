\documentclass[12pt]{article}
\special{papersize=3in,5in}

%-------------------------------------------------------
% PACKAGES

\usepackage[utf8]{inputenc}
\usepackage{amssymb,amsmath}
\usepackage{natbib}
\usepackage{outlines}
\pagestyle{empty}
\setlength{\parindent}{0in}

%-------------------------------------------------------
% CONFIG

\author{Francisco Alves}
\title{Government at Risk}
%\date{Heij et al 6.5 a-d}

%-------------------------------------------------------
% BODY

\begin{document}

\maketitle

%-----------------------------------------------

There are at least three dimensions that make governments at risk a problem worth closer scrutiny. 

First, fiscal risks pose a possible threath to fiscal sustainability simply because of the fact that contigent liabilities ``tend to remain outside the framework of conventional public finance analysis and institutions.'' \citep[p. 1]{brixi2002} The problem in this case are those that ultimately can be associated with an unsustainable fiscal position (not necesseraly slow economic growth).

Second, ``if the mechanisms for information disclosure are weak and market institutions not well developed, the risks assumed by government generate a bias in the behaviour of eocnomic agents and moral hazard in the markets, and thus work against development even before they are realized'' \citep[p. 2]{brixi2002}. This is perhaps better though as a game theorethic problem of assymetric information.

Finally, ``because contigent liabilities often grow from fiscal opportunism, when policymakers seek to hide the real fiscal cost of their decisions and to reduce the reported budget deficit''  \citep[p. 2]{brixi2002} we sometimes can have a fiscal politics problem, which again is better understood as a strategic interaction problem.

What is nevertheless true is that the second and third dimensions of fiscal risks still bring fiscal challenges, but it's important to recognize that the mechanisms to face them are more likely then not to be different than the toolkit used to simply deal with assessing the sustainability of fiscal risks.

To drive home the point, the argument is that the fiscal sustainability problem is akin to a measurement problem. This problem is difficult ``because it requires government to enter uncharted fiscal territory, where analytical frameworks are sometimes difficult to apply, accouting standards are underdeveloped or poorly enforced, and the data are inadequate or hidden from public scrutiny.'' \citep[p. 2]{brixi2002}. Although this are all difficult problems, we should never loose sight that the other dimensions related to the strategic nature of the economic interaction taking place is far more wicked.



%-------------------------------------------------------
% REFERENCES
\bibliographystyle{agsm}
{\footnotesize
\bibliography{../../references.bib}}

\end{document}

