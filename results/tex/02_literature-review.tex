\chapter{Literature Review}
\label{sec:literature-review}

As noted by \citet{baldacci2011b}, the literature that aims to build models that can provide early warning signals of fiscal sustainability problems tend to differ with respect to three major characteristics: the definition of the crisis event; the statistical methodology employed; and the set of explanatory variables used. Broadly following this differences, in section \ref{sec:fiscal-distress} we review theoretical definitions of fiscal sustainability in order to propose a definition of fiscal distress\footnote{For the purposes of this study, fiscal distress, fiscal stress, fiscal crisis, and debt crisis will be used as synonymous} event that will be used in this study. In section \ref{sec:ews} we present a general formulation of the early warning system problem and the methodological approach pursued in this study. Regarding the set of explanatory variables, since the main objective of this study is to assess the significance of the fiscal indicators used by the National Treasury Secretariat in their mandate to evaluate the payment capacity of subnational governments, the explanatory variables are already defined. Therefore we let the presentation of the fiscal indicators for section \ref{sec:data} coupled with the exploratory analysis of the data.

% in section \ref{sec:brazil} we provide a brief overview of the evolution of fiscal distress events of SGNs that happened in the Brazilian economy since the 80s and 90s and the reforms initiatives that followed in order to characterize the present rules regarding borrowing by SGNs.

% \footnote{Although not used as much in the economic literature, early warning signals models are also called predictive models in the statistical and machine learning literature. The important difference is that there is no preoccupation with causal inference in this type of exercise}

\section{Fiscal Distress}
\label{sec:fiscal-distress}

The focus of this study will be on the empirical analysis of the main determinants of fiscal distress episodes in subnational governments in Brazil in the 2008-2016 period. Nevertheless, it is important a brief theoretical review of the most used concepts of fiscal sustainability so that an appropriate characterization of what fiscal distress event entails, or, and perhaps even more importantly, what it do not entail.

There are a few distinct ways in which fiscal sustainability is defined in the economic literature. Although they all in some way or another alludes to the more general concept of sustainability, understood here as a process that ``meets the needs of the present without compromising the ability of future generations to meet their own needs'' \citep{un1987}, the differences are important for the interpretation of the empirical results and should be emphasized at the outset.

A more narrow definition of fiscal sustainability equates it to solvency, that is, the ability of an entity, in the case of this study a subnational government, to make payments in order to service its debt obligations on the due date. Although certainly useful for corporations, this definition is far too stringent when applied to governments. The reasoning is not as much because governments do not actually default on its debt, but, as pointed out by \citet{burnside2005}, when it is clear that a policy mix is unsustainable, governments tend to take remedial actions in order to avoid an outright default. A more useful definition of fiscal sustainability takes this into account and can be equated ``to a government’s ability to indefinitely maintain the same set of policies while remaining solvent'' \citep[pg.11]{burnside2005}. Therefore it makes sense to speak of an unsustainable policy mix even though a default on the debt never actually took place. For the purposes of this study, both an explicit default on the debt and an unsustainable policy mix should be considered an indicator of a fiscal distress event. 

It is worth pointing out that in the definitions given so far, no distinction was made between solvency and liquidity problems. A country may believe itself to be solvent, but it still faces problems in meeting its obligations because of cash flows problems. The reason we don't pursue this distinction is less based on the fact that there are no theoretical differences between the two concepts, but because the empirical consequences of both difficulties are likely to be observationally equivalent \citep[pg.89]{chuhan2005}. Both solvency and liquidity are related to an entity ability to pay. However, especially because of the absence of clearly defined rules for bankruptcy in the public sector, the government willingness to pay also becomes important, in a tradition that goes back at least to \citet{eaton1981}. Again, however, for the purposes of this study, both are fiscal crisis episodes stemming from ability or willingness to pay are likely to be observationally equivalent.

One of the uses of fiscal sustainability that will not be used in this study is related to the costs, in terms of economic efficiency or growth, related to a given combination of fiscal and monetary policy. Although this use is suggested by \citet{burnside2005}, we shall make no claim in this study related to it.

After looking at the theoretical definitions of fiscal sustainability, its worth to take a closer look into how previous literature on EWS defined the debt crisis episodes for sovereign governments. \citet{manasse2003} defines a country to be in a debt crisis if either the government fails to meet principal or interest payment on external obligation on the due date as classified by Standard \& Poor's or if it receives a nonconcessional IMF loan in excess of 100 percent of its quota. \citet{fuertes2007} considers a country to be in default in a given year if the arrears increase over a threshold percentage of external debt and a rescheduling agreement is reached in which the amount of debt rescheduled exceeds the decrease in the arrears stock. \citet{baldacci2011b} uses a more general definition and considers a fiscal crisis not only the episodes of debt default or restructuring and recourse to exceptional financing, but also an implicit default crystallized in high inflation rates and a deterioration in market access measured by high bond yields pressures (where high is those that yield spreads that are more than two standard deviations away from the mean).

Although the empirical literature is of limited applicability in giving operational guidance in defining fiscal crisis episodes for this study because of differences in sovereign and subnational governments, it is possible to see that the theoretical elements discussed of solvency and liquidity are present. 

\section{Early Warning Systems}
\label{sec:ews}

Before we delve into the different methodological approaches used in the literature in early warning systems, we need a general framework to capture the purpose of an EWS model. Following \citet{fuertes2007}, we denote by $d_{it}$ the dummy variable that equals 1 if the state $i$ had a crisis event in period $t$ and 0 otherwise. Since the objective is to signal crisis in advance, the dependent variable $y_{it}$ of interest is forward-looking in nature, and it takes the value of one if a crisis happens during an $h$ time horizon, that is
\begin{equation}
y_{it} = \begin{cases} 
      1 & \text{if } d_{i,t+k} = 1 \text{ for any } k = 0, \dots, h-1\\
      0 & \text{otherwise}
   \end{cases}
\end{equation}

However, it is important to remark that the information set\footnote{The information set, usually denoted by $\Omega_t$ consists of the set of all potential explanatory variables that could be included in a regression model.} for predicting $y_{it}$ is that available at time $t-1$. To fix ideas, let the explanatory variable be the primary balance as a ratio of GDP and let $t = 2016$. A prediction $\hat{y}_{it} = 1$ from an EWS with horizon $h = 1$ implies that using the primary balance as a ratio to GDP from 2015 backwards, the model is signalling a potential crisis in 2016 for state $i$. Similarly, if $h = 2$, the model is signalling a potential crisis in 2016 or 2017. Note that again only the primary balance as a ratio to GDP from 2015 backward is used to make this prediction. 

In a more general notation, if we let $\vx$ denote the explanatory variables included in the model, and the past of $\vx$ for state $i$ as $\mX_{i,t-1} = \{\vx_{i,t-1}, \vx_{i,t-2}, \dots \}$, the prediction problem of an EWS with a given horizon $h$ is given by
\begin{equation}
\label{eq:classifier}
y_{it} = f(\mX_{i,t-1})
\end{equation}

In practice using the whole past $\mX_{i,t-1}$ of the explanatory variables is not possible, and we will follow other studies using only the last period variables, that is, $\vx_{i, t-1}$. This is not problematic as long as there are stock variables that can capture the effects of flows from previous periods.

For estimating the function $f$ in (\ref{eq:classifier}),  there are two major approaches in the literature on early warning systems. The ``indicators'' or ``signaling'' approach and the multivariate regression analysis approach \citep[pg. 5]{baldacci2011b}. The first approach belongs to the class of non-parametric methods. Their essential characteristic is that they do not assume a specific functional form for $f$, and simply try to estimate a smooth relationship between the explanatory and dependent variables. The second approach belongs to the class of parametric methods. In this case, a specific functional form for $f$ is assumed, and, with this knowledge at hand, the relevant parameters are estimated\footnote{\citet{james2013} is a nice introduction and overview of statistical learning techniques, both parametric and non-parametric. \citet{hastie2009} is a more advanced and complete treatment.}. In this study, we will take the latter approach. More specifically, we will make use of a limited dependent variable model know as logit regression\footnote{The specific characteristics of the model employed are discussed in section \ref{sec:econometric-model}}. 

As noted by \citet{baldacci2011b}, the main reason for using the multivariate approach in this study is the easily available null hypothesis significance tests that can be conducted to assess the statistical significance of both individual variables and collection of variables. This allows for a clean way to attain one of the objectives of this study, that is, to compare and contrast the current and the newly set of fiscal indicators used by the National Treasury Secretariat to assess the payment capacity of SGNs in Brazil.

% can we do NHST in non-parametric tests?

% \section{Fiscal Indicators}
% \label{sec:fiscal-indicators}

 % This section aims not only to present the fiscal indicators, but also to take a step back and give a brief overview of the broader institutional context that characterizes the Brazilian fiscal framework and that culminated in the current legislation regarding subnational borrowing in Brazil. 

% - Brazil Subnational Debt Restructuring of the 1990s
% - Fiscal Responsibility Law
% - 2016 Debt Relief Plan for the States and the Federal District (PLP 257/2016, LC 156/2016)
% - 2017 Fiscal Recovery Regime for the States and the Federal District (PLC 39/2017, LC 159/2017)
% - Payment Capacity Evaluation Methodology

% Although the above characterization could be employed to describe the recent economic history of several countries, Brazil might as well be the canonical example. A federation with one Federal District, 26 states, and 5,570 municipalities, Brazil experienced in the 80s and 90s repeated fiscal crisis of subnational entities with three rounds of debt restructurings. These episodes severely threatened the success of several macroeconomic stabilization programs that hoped to control the three-to-four-digit annual inflation rates that wreak havoc the Brazilian economy from 1980 through 1994. It was only with the success of the Real Plan in fighting the hyperinflation, and with a more comprehensive debt restructuring that aimed to correct the underlying SNG fiscal problems, that the political and economic conditions enabled for a complete reformulation of the fiscal institutions in place, culminating with the publication of a Fiscal Responsibility Law (FRL) in 2000 \citep[p. 34-35]{manoel2013}. 

% The FRL was a comprehensive law that promoted several changes in the institutions related to the Brazilian budgetary process. However, for the purposes of this study, the focus will be on the controls that were put forth for SGNs borrowings. In Brazil, if the central government must offer a guarantee for an individual borrowing operation, then the Finance Ministry, through the National Treasury Secretariat (NTS), must assess that entity ``payment capacity'' before the operation is authorized. The current methodology employed by NTS was enacted in 2012 and makes use of several fiscal indicators to classify the SGNs entities in different credit classifications, similar in spirit to the process adopted by rating agencies in giving credit ratings. However, this methodology is currently being revised \footnote{The NTS published the new set fiscal indicators and opened the methodology for a public revision process in the period of 10/05/2017 to 30/06/2017. The material related to this revision is available at \url{http://www.tesouro.fazenda.gov.br/sistemagarantiauniao}}, as the NTS is looking for an alternative suite of fiscal indicators that are more transparent and more easily calculated. In this context, the assessement of the statistical and practical significance of both the current and the newly set of fiscal indicators used by the National Treasury Secretariat to assess the fiscal sustainability of subnational governments in Brazil is both necessary and timely. 
