\chapter{Conclusion}
\label{sec:conclusion}

The purpose of the current study was to evaluate the statistical and practical significance of both the current and the newly proposed fiscal indicators used by the National Treasury Secretariat (NTS) to assess the payment capacity of subnational governments in Brazil. The current methodology enacted in 2012 by the NTS was deemed to be overly complicated in general and also based on a set of fiscal indicators that could potentially be simpler in terms of dimensionality without harm in terms of predictive performance. This view was justified on the basis of the correlation between the fiscal indicators. %add citation

This study has identified that the new methodology that uses only $3$ ratios instead of the $8$ of the current methodology is in fact statistically superior in terms of the likelihood of the given data. This superior performance is mostly attributed to the ratio Current liabilities / Cash and cash equivalents who proved itself to be significant in all specifications employed in this study. 

Expressing these results in the language of the concepts related to fiscal sustainability discussed in section \ref{sec:fiscal-distress}, we can say that empirically, the willingness and ability to pay of an SGN in Brazil appear to be mostly explained by its liquidity position. Solvency ratios appear to be only instrumental for this explanation, in the sense that they matter only to the extent that indirectly they influence the trajectory of liquidity ratios. To exemplify, high debt ratios to net current revenue can only explain fiscal crisis episodes in Brazil to the extent that debt service obligations generate higher current liabilities. 

The research has also shown that for the sample at hand, the fiscal indicators proposed in the new methodology proposed by the NTS could maybe fruitfully include Gross investment in nonfinancial assets / Total expenditure. However, this result appears to be more related to the observed behavior of Rio de Janeiro, who, even in financial difficulties, kept investment rates high because of the commitments made with the 2016 Summer Olympics.

There are four major sources of weakness in this study, all in some way or another related to data constraints. First, we did not allow for the presence of unobservable heterogeneity between states, by using, for example, a fixed effects logit model. The reason for this is simply that in fixed effect estimation all the observations corresponding to states that did not face a crisis episode in the sample horizon would be dropped out of the likelihood function, leaving only the observations of Rio de Janeiro, Rio Grande do Sul and Minas Gerais. 

The second source of weakness is related to the definition of fiscal crisis episode adopted. The enactment of a decree of financial calamity is a political process that a given state might not participate because it does not align with the political calculus of the politicians involved in this decision. Although this sounds like a tautology, it is especially important in the current case. The reason is that the most common interpretation of the Brazilian legal system, shared by the Ministry of Finance\footnote{\url{http://g1.globo.com/bom-dia-brasil/noticia/2016/12/calamidade-financeira-de-estados-nao-e-reconhecida-pelo-governo.html}}, says that the decree of calamity should be restricted to natural disasters, and therefore the benefits, such as the possibility to delay payments to creditors and to bypass some legal requirements for procurement and budgeting process, is not valid under ``financial calamity''. A more reliable definition of fiscal crisis episode would need to make use of arrears data that still do not exist for SGN's in Brazil.

A third source of weakness comes from the 2008-2016 horizon employed. Brazil experienced in the 80s and 90s repeated fiscal crisis of subnational entities with three rounds of debt restructurings. These should clearly be considered a fiscal crisis episode. However, the majority of the explanatory variables used in this study were fiscal variables and indicators derived from the datasets provided by the National Treasury Secretariat. Although, the original period covered by the data published by NTS extends from 1986 through 2016, totalling 31 years of data, events like hyperinflation, change of currencies, change in fiscal reporting and budget classifications and no tracking of stock variables makes the process of compiling a cleaned and consistent dataset a research enterprise of its own.

Finally, because there were only three fiscal crisis episodes in the sample that happened in the same year, it was not possible to look into out of sample forecast accuracy measures, which are ultimately the final yardstick by which EWS models should be judged. \citep{berg2005}.

Putting the need for the collection of a more comprehensive dataset of fiscal variables on SGN aside, the cited weakness of this study are useful alternatives for future work. Additionally, because most of the EWS literature is currently focused on sovereign countries, studies that use different strategies in the three areas that tend to differentiate early warning systems models, namely, the definition of the crisis event, the statistical methodology employed, and the set of explanatory variables, but applied to SGNs, would be a welcome addition to the literature.

