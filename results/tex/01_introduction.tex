\chapter{Introduction} 

There is a clear consensus both in the academic literature and in the policy debate that sound management of fiscal risks is essential for both fiscal sustainability and macroeconomic stability of sovereign countries \citep{brixi2002, kopits2014, imf2016}. One of the classes of fiscal risks that were particularly neglected in the usual fiscal sustainability analysis is that related to contingent liabilities, that is, liabilities whose occurrences depends on the outcome of an uncertain event \citep{brixi2002}. A second important development in recent years is that with the increased decentralization in the provision of public services, the contingent liabilities, both explicit and implicit, arising from subnational governments (SNGs) are increasingly relevant for fiscal sustainability analysis. This relevance is higher the more spending and taxation powers are given to the subnational entities in the fiscal framework currently in place in a given country.

Although the above characterization could be employed to describe the recent economic history of several countries, Brazil might as well be the canonical example. A federation with one Federal District, 26 states, and 5,570 municipalities, Brazil experienced in the 80s and 90s repeated fiscal crisis of subnational entities with three rounds of debt restructurings. These episodes severely threatened the success of several macroeconomic stabilization programs that hoped to control the three-to-four-digit annual inflation rates that wreak havoc the Brazilian economy from 1980 through 1994. It was only with the success of the Real Plan in fighting the hyperinflation, and with a more comprehensive debt restructuring that aimed to correct the underlying SNG fiscal problems, that the political and economic conditions enabled for a complete reformulation of the fiscal institutions in place, culminating with the publication of a Fiscal Responsibility Law (FRL) in 2000 \citep[p. 34-35]{manoel2013}. 

The FRL was a comprehensive law that promoted several changes in the institutions related to the Brazilian budgetary process. However, for the purposes of this study, the focus will be on the controls that were put forth for SGNs borrowings. Following the typology of borrowing controls used by \citet{ahmad2005} and first proposed by \citet{ter1997}, Brazil has adopted both a rules-based control and an administrative control for SNG borrowing. In a rules-based control, a fiscal rule is imposed that directly constrains the SNG ability to borrow. In Brazil, the FRL adopted both a golden rule and a debt ceiling rule. In an administrative control, the central government has some form of direct control over the SGN borrowing. In Brazil, if the central government must offer a guarantee for an individual borrowing operation, then the Finance Ministry, through the National Treasury Secretariat (NTS), must assess that entity ``payment capacity'' before the operation is authorized. The current methodology employed by NTS was enacted in 2012 and makes use of several fiscal indicators to classify the SGNs entities in different credit classifications, similar in spirit to the process adopted by rating agencies in giving credit ratings. However, this methodology is currently being revised \footnote{The NTS published the new set fiscal indicators and opened the methodology for a public revision process in the period of 10/05/2017 to 30/06/2017. The material related to this revision is available at \url{http://www.tesouro.fazenda.gov.br/sistemagarantiauniao}}, as the NTS is looking for an alternative suite of fiscal indicators that are more transparent and more easily calculated. In this context, the assessement of the statistical and practical significance of both the current and the newly set of fiscal indicators used by the National Treasury Secretariat to assess the fiscal sustainability of subnational governments in Brazil is both necessary and timely. 

% THIS IS THE MAIN OBJECTIVE OF THE STUDY ALTHOUG IS NOT EXPLICIT STATED

The strand of literature on Early Warning Systems (EWS) for economic crisis are particularly relevant for this endeavour. Following the seminal work by \citet{kaminsky1998} and \citet{berg1999} on currency crisis, it was not long until fiscal crisis were also tackled \citep{manasse2003, fuertes2007, baldacci2011b, berti2012, dawood2017}. In this study, the focus will be on what is usually called the parametric approach, that makes use of limited dependent variables models, such as probit and logit. There are three main contributions. First, as usually true in most countries, disaggregate fiscal data on regional governments tends to have a much lower quality and standardization then those available for central governments\footnote{Although there are several initiatives in place in Brazil to modernize the information available for SNGs, in the fiscal transparency evaluation conducted by the IMF in Brazil completed in June 2016 and published in May 2017 it is noted that ``Weaknesses in fiscal reporting also undermine the ability to assess the fiscal position and risks. Not all states and municipalities comply with their reporting obligations, and information on subnational finances is generally less timely and comprehensive than information on the central governments. \citep[p. 62]{imf2017}''}. This study aims to compile, consolidate and make available on machine readable format regional discriminated data on the fiscal variables that are needed to compute the fiscal indicators used by the NTS for payment capacity evaluation. To the best of the author's knowledge, this database is inexistent today. The second contribution is the expansion of the Early Warning System literature for dealing with subnational governments. Most of the studies focus on sovereign governments, but, as noted by \citet{ianchovichina2007}, SGNs are sufficiently different from sovereign governments and demand separate analysis. The third and final contribution is that by recognizing a fiscal crisis as a rare event as suggested by \citet{king2001a}, that is, a binary dependent variable with dozens to thousands of times fewer ones than zeros, we make use of the bias-reduction method first proposed by \citet{firth1993} and implemented by \citet{brglm} in the R-package \texttt{brglm}.

The remainder of this study is organized as follows. Section \ref{sec:literature-review} presents a brief literature review on the main characteristics of early warning systems for economic crisis and also give a general overview of the institutional context under which the National Treasury Secretariat conducts its payment capacity evaluations. Section \ref{sec:methods} presents a descriptive analysis of the data and also outline the econometric model used. Section \ref{sec:results} presents the empirical results and section \ref{sec:conclusion} summarizes our conclusions.