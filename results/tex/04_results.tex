\chapter{Results}
\label{sec:results}

Before presenting the econometric results, a word of caution is in order. It is important to keep in mind that all the results rest on the definition of fiscal distress event, which, in this study, produced only $3$ crisis episodes across the sample of $243$ observations. As discussed in section \ref{sec:econometric-model}, this raises issues of bias in the estimates and problems related to complete and quasi-complete separation. Although we are making use of a bias reduction method proposed by \citet{firth1993} and implemented by \citet{brglm}, by no means this puts a definitive end to concerns related to the robustness of the results.

The general objective of this study is to test the statistical significance of the fiscal indicators used the by the National Treasury Secretariat (NTS) in payment capacity evaluations. However, because the NTS is currently revising the methodology employed in those evaluations, two similar but still different strategies could profitably be pursued in order to assess this set of fiscal indicators. The first would began by considering all fiscal ratios ever defined by the NTS\footnote{That is to say the fiscal indicators defined in the Finance Minister Decree nº 306, 10/09/2012 and those in the newly proposed methodology available at \url{http://www.tesouro.fazenda.gov.br/sistemagarantiauniao}} and from that set try to exclude variables that are not significant. Following \citet{kennedy2008}, this would be a ``testing down'' specification approach. The second strategy would begun by taking into consideration only the fiscal indicators that the NTS is proposing to become the new standard, and from this set test if they could frutifully be augmented by other fiscal ratios. Again, in the sense suggested by \citet{kennedy2008}, this would be a ``testing up'' specification strategy.

In this study, we followed both approaches. More specifically, in the ``testing down'' approach, we began with a model that includes all explanatory variables. From this full model, we considered two reduced nested models that were formed based on the fiscal indicators present in each payment capacity methodology. This allows us to test for the joint significance of several explanatory variables and directly compare the significance of each methodology. In the ``testing up'' approach, the base model uses as explanatory variables only the $3$ fiscal ratios proposed in the new payment capacity evaluation. We then run individual regressions against each of the fiscal variables set forth in the Finance Minister Decree nº 306, 10/09/2012. Finally, all significant variables of the second stage are then added one by one to the base model, and we conduct tests for the significance and look for changes in sign and/or significance of the base variables. This strategy allows us to better grasp if the proposed set of fiscal indicators of the new methodology could usefully be expanded with previously used ratios while minimizing the number of tests conducted. We don't ``test down'' the full model because it is not clear what criteria should be used to exclude a given variable at each stage without either going through all combinations or making more or less arbitrary choices.

Two last remarks are in order before we delve into the results. First, the ratio Gross Debt / Net current revenue is included in both methodologies but obviously will be included only once in the full model. Similarly, the calculation rule of the ratio Current fiscal balance / Current revenue was changed in the new methodology to Current expenses / Current revenue. However, since they convey the same information, in order to avoid multicollinearity issues, only the ratio Current expenses / Current revenue from the newly proposed methodology will be used for estimation purposes. Second, we follow the suggestion given in \citet{gelman2008} and scale all fiscal indicators with a division by two standard deviations, that is, $z_{i,t-1} = x_{i, t-1} - mean(x) / 2 \cdot sd(x)$. The interpretation of the regression coefficients is equal to the mean $\pm 1$ standard deviation, the same comparison of possible untransformed binary predictors. To say it differently, the coefficients can now be interpretable as changes from low to high values of the explanatory variable of interest.

\subsection*{Testing down}

Table \ref{tbl:model_coef_test_down} presents the regression estimates for the three first models of interest. Model 1 is the full model and includes all fiscal indicators used by the NTS, be in the current or the new methodology. We also estimate two nested models that correspond to particular restrictions applied to the full model. Model 2 corresponds to a restricted model in which all variables of the current methodology are set to zero. Model 3 on the other hand, corresponds to the restricted model in which all variables of the new methodology are set to zero. Although, in this case, only the ratio Current liabilities / Cash and cash equivalents is set to zero.

It is worth reminding the reader that, contrary to linear models, the coefficients in non-linear models can't be interpreted as marginal effects, although we can interpret both the sign and the relative magnitude of the relative magnitude of the coefficients. 

In model 1, both Current liabilities / Cash and cash equivalents and Gross investment in nonfinancial assets / Total expenditure were significant at a 5\% level. However, the sign in the latter was somewhat surprising. We would expect that states that were on the verge of a crisis would reduce their investment rate because of the discretionary nature of this type of expenditure. Therefore, in a predictive sense, a higher investment rate would translate into a smaller probability of a fiscal crisis, the opposite of what was found in model 1. In model 3 the sign is negative as expected, but the variable was no longer significant. One possible explanation is related to a large amount of investment spending undertaken by Rio de Janeiro for the 2016 Summer Olympics, even when they were already financially constrained. In model 2 only Current liabilities / Cash and cash equivalents was significant at a 5\% level. In model 3, that excludes only Current liabilities / Cash and cash equivalents with respect to model 1, several variables were significant. Gross debt / Net current revenue and Primary balance / Debt service were significant at the 5\% level while Current expenses / Current revenue and Compensation of employees / Net current revenue at the 10\% level. Because of this drastic change of the significant explanatory variables in going from model 1 to model 3, it does raises concerns about an omitted variable bias in model 3.

In order to better grasp which model is in some sense better, we can conduct a likelihood-ratio test described in section \ref{sec:econometric-model}. First, likely due to the bias-reduction method employed, the log-likelihood for model 2 is, in fact, \textbf{greather} than the log-likelihood for the unrestricted model 1, meaning that there was a \textbf{likelihood gain} moving from imposing the parameter restriction. For model 3, the test statistic is $LR = 2 \cdot (-5.964 + 8.246) = 4.56$. Compared with a $\chi^2$ with one degree of freedom, the p-value is equal to $0.033$. We, therefore, reject the null hypothesis at a 5\% significance level, meaning that the loss of likelihood in imposing the restriction is likely different from zero and the unrestricted model (model 1) should be considered the best one. Together, these results indicate that in choosing between the information set of the current versus the newly proposed methodology, the newly proposed methodology appears to be the better choice.


% Table created by stargazer v.5.2 by Marek Hlavac, Harvard University. E-mail: hlavac at fas.harvard.edu
% Date and time: Wed, Aug 02, 2017 - 13:13:54
% Requires LaTeX packages: dcolumn 
\begin{table}[!htbp] \centering 
  \caption{Regression results for binary dependent variable - Testing down} 
  \resizebox{\textwidth}{!}{%
  \label{tbl:model_coef_test_down} 
\begin{tabular}{@{\extracolsep{5pt}}lD{.}{.}{-3} D{.}{.}{-3} D{.}{.}{-3} } 
\\[-1.8ex]\hline 
\hline \\[-1.8ex] 
\\[-1.8ex] & \multicolumn{1}{c}{(1)} & \multicolumn{1}{c}{(2)} & \multicolumn{1}{c}{(3)}\\ 
\hline \\[-1.8ex] 
 Gross debt / Net current revenue & 1.674 & 1.048 & 4.089^{**} \\ 
  & (1.283) & (1.254) & (1.785) \\ 
  & & & \\ 
 Current expenses / Current revenue & -1.754 & -1.110 & -2.805^{*} \\ 
  & (1.288) & (1.278) & (1.491) \\ 
  & & & \\ 
 Current liabilities / Cash and cash equivalents & 5.001^{***} & 3.531^{**} &  \\ 
  & (1.932) & (1.396) &  \\ 
  & & & \\ 
 Debt service / Net current revenue & -0.882 &  & 0.444 \\ 
  & (1.341) &  & (1.239) \\ 
  & & & \\ 
 Primary balance / Debt service & 0.704 &  & -4.482^{**} \\ 
  & (0.717) &  & (2.027) \\ 
  & & & \\ 
 Compensation of employees / Net current revenue & 0.876 &  & 1.976^{*} \\ 
  & (0.943) &  & (1.069) \\ 
  & & & \\ 
 Gross investment in nonfinancial assets / Total expenditure & 2.926^{**} &  & -1.306 \\ 
  & (1.472) &  & (1.459) \\ 
  & & & \\ 
 Social contributions / Social benefits & 0.482 &  & 0.731 \\ 
  & (0.812) &  & (0.714) \\ 
  & & & \\ 
 Tax revenues / (Current expenses + Principal payments) & -0.051 &  & -0.087 \\ 
  & (1.523) &  & (1.375) \\ 
  & & & \\ 
 Constant & -5.834^{***} & -5.813^{***} & -6.098^{***} \\ 
  & (1.349) & (1.280) & (1.434) \\ 
  & & & \\ 
\hline \\[-1.8ex] 
Observations & \multicolumn{1}{c}{216} & \multicolumn{1}{c}{216} & \multicolumn{1}{c}{216} \\ 
Log Likelihood & \multicolumn{1}{c}{-5.964} & \multicolumn{1}{c}{-5.804} & \multicolumn{1}{c}{-8.246} \\ 
Akaike Inf. Crit. & \multicolumn{1}{c}{31.929} & \multicolumn{1}{c}{19.608} & \multicolumn{1}{c}{34.492} \\ 
\hline 
\hline \\[-1.8ex] 
\end{tabular}%
} 
\fnote{Notes: $^{*}$p$<$0.1; $^{**}$p$<$0.05; $^{***}$p$<$0.01 and standard errors reported in parentheses\\
       1) Model 1 is the full model, model 2 is the reduced model using only the fiscal indicators of the newly proposed methodology and model 3 is the reduced model using only the fiscal indicators given in the Finance Minister Decree nº 306, 10/09/2012}
\end{table} 


\clearpage

\subsection*{Testing up}

Table \ref{tbl:model_coef_test_up} presents the regression estimates for the ``testing up'' approach. Model 1 (Model 2 in table \ref{tbl:model_coef_test_down}) is the base model that uses the fiscal ratios proposed in the new payment capacity evaluation as explanatory variables. We first ran individual simple regressions against each of the fiscal variables set forth in the Finance Minister Decree nº 306, 10/09/2012. The ratios Debt service / Net current revenue, Compensation of employees / Net current revenue and Gross investment in nonfinancial assets / Total expenditure were significant in the individual regression. We then added the ratios to the base model. This is reported in models 2 through 4 in table \ref{tbl:model_coef_test_up}. Debt service / Net current revenue (Model 2) was not significant and didn't change the sign of the other explanatory variables. Compensation of employees / Net current revenue (Model 3) was also not significant, but with its inclusion in the model Current expenses / Current revenue was significant at a 5\% level. The sign, however, was somewhat puzzling. It indicates that holding compensation of employees fixed, lower savings reduces the probability of a crisis. There are no readily available explanations for this result except for the observation that Compensation of employees is the most significant component of current expenses in all states. Gross investment in nonfinancial assets / Total expenditure was (Model 4) was not only significant at the 10\% level, but made Gross debt / Net current revenue significant as well at the same level. The sign was again positive, implying that higher investment is correlated with higher probability of a fiscal crisis. 

We finally add Compensation of employees / Net current revenue and Gross investment in nonfinancial assets / Total expenditure to the base model (Model 5) giving what can be considered the final specification of the ``testing up'' strategy. In this model Current liabilities / Cash and cash equivalents and Gross investment in nonfinancial assets / Total expenditure were significant at the 5\% level. We again conduct a likelihood ratio test to compare the unrestricted model (model 5) with the restricted one (model 1). The test statistic is $LR = 2 \cdot (-4.063 + 5.804) = 3.48$. Compared with a $\chi^2$ with two degrees of freedom, the p-value is equal to $0.175$. We, therefore, fail to reject the null hypothesis. The interpretation is that it might be the case that the loss of likelihood by excluding the two variables from the unrestricted model is not different from zero. Again this indicates that restricting the information set to the fiscal indicators of the newly proposed methodology appears to be a reasonable choice.


% Table created by stargazer v.5.2 by Marek Hlavac, Harvard University. E-mail: hlavac at fas.harvard.edu
% Date and time: Sat, Aug 05, 2017 - 13:37:17
% Requires LaTeX packages: dcolumn 
\begin{table}[!htbp] \centering 
  \caption{Regression results for binary dependent variable - Testing up} 
  \label{tbl:model_coef_test_up} 
\resizebox{\textwidth}{!}{%
\begin{tabular}{@{\extracolsep{5pt}}lD{.}{.}{-3} D{.}{.}{-3} D{.}{.}{-3} D{.}{.}{-3} D{.}{.}{-3} } 
\\[-1.8ex]\hline 
\hline \\[-1.8ex] 
\\[-1.8ex] & \multicolumn{1}{c}{(1)} & \multicolumn{1}{c}{(2)} & \multicolumn{1}{c}{(3)} & \multicolumn{1}{c}{(4)} & \multicolumn{1}{c}{(5)}\\ 
\hline \\[-1.8ex] 
 Gross debt / Net current revenue & 1.048 & 0.806 & 1.451 & 6.002^{*} & 2.465 \\ 
  & (1.254) & (1.170) & (1.357) & (3.631) & (1.591) \\ 
  & & & & & \\ 
 Current expenses / Current revenue & -1.110 & -1.625 & -3.203^{**} & 0.112 & -1.880 \\ 
  & (1.278) & (1.340) & (1.621) & (1.562) & (1.475) \\ 
  & & & & & \\ 
 Current liabilities / Cash and cash equivalents & 3.531^{**} & 4.177^{**} & 4.237^{**} & 9.076^{*} & 4.617^{***} \\ 
  & (1.396) & (1.758) & (1.784) & (4.785) & (1.767) \\ 
  & & & & & \\ 
 Debt service / Net current revenue &  & -0.576 &  &  &  \\ 
  &  & (1.207) &  &  &  \\ 
  & & & & & \\ 
 Compensation of employees / Net current revenue &  &  & 1.586 &  & 1.405 \\ 
  &  &  & (1.076) &  & (1.098) \\ 
  & & & & & \\ 
 Gross investment in nonfinancial assets / Total expenditure &  &  &  & 8.737^{*} & 3.566^{**} \\ 
  &  &  &  & (4.782) & (1.686) \\ 
  & & & & & \\ 
 Constant & -5.813^{***} & -5.580^{***} & -6.433^{***} & -14.206^{**} & -6.973^{***} \\ 
  & (1.280) & (1.187) & (1.669) & (6.884) & (1.937) \\ 
  & & & & & \\ 
\hline \\[-1.8ex] 
Observations & \multicolumn{1}{c}{216} & \multicolumn{1}{c}{216} & \multicolumn{1}{c}{216} & \multicolumn{1}{c}{216} & \multicolumn{1}{c}{216} \\ 
Log Likelihood & \multicolumn{1}{c}{-5.804} & \multicolumn{1}{c}{-6.039} & \multicolumn{1}{c}{-4.609} & \multicolumn{1}{c}{-2.505} & \multicolumn{1}{c}{-4.063} \\ 
Akaike Inf. Crit. & \multicolumn{1}{c}{19.608} & \multicolumn{1}{c}{22.078} & \multicolumn{1}{c}{19.218} & \multicolumn{1}{c}{15.010} & \multicolumn{1}{c}{20.127} \\ 
\hline 
\hline \\[-1.8ex] 
\end{tabular}%
} 
\fnote{Notes: $^{*}$p$<$0.1; $^{**}$p$<$0.05; $^{***}$p$<$0.01 and standard errors reported in parentheses\\
       1) Model 1 is the base model using the variables proposed in the new methodology. Models 2 through 4 add the fiscal indicators defined on the Finance Minister Decree nº 306, 10/09/2012 that were significant at the 5\% significant level on the individual regressions}
\end{table} 


\clearpage