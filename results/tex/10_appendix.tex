\appendix
\chapter{Data Definitions and Sources}
\label{appendix:data-source}

The dataset used in this study is available online at 

The majority of the explanatory variables used in this study will be fiscal variables and indicators derived from the datasets provided by the National Treasury Secretariat (STN). This section aim is to address the main data gaps that prevent the use of some specific years in the empirical analysis of this study. The original period covered by the data published by STN extends through 1986 through 2016, totalling 31 years of data. 

The first issue arises in the period from 1980 through 1994 when Brazil was facing a hyperinflation crisis and used several different currencies. Taking into considerations the difficulty in compiling accurate statistics during a period in which prices are changing so rapidly, the possible relevant period is shortened to 1995-2016. 

The second issue is related to the change in fiscal reporting practices brought about by the the Fiscal Responsibility Law in 2000-05-04 and the change in the Economic Classification of expenditures in 2001-05-04. Before this period, only flow variables are published by the STN, and they are not fully consistent with the more recent series. Taking into account the relevance of stock variables for fiscal analysis, the period is again shortened to 2002-2016.

The final issue is related to the conditions surrounding the bailouts provided by the National government. The states were allowed a cap on the maximum amount of debt service that they needed to pay in any given year in terms of their net revenue (Receita Liquida Real), with the difference being incorporated into the debt stock. However, instead of using an accrual basis of accounting, only the interest paid was registered. The central back publishes statistics on an accrual basis for regional governments, but they only begin in 2008. Taking into consideration the relevance of this variable, the period is again shortened to 2008-2016.

\begin{table}[!ht] \centering 
  \caption{Compiled Fiscal Variables} 
  \label{tbl:fares-general} 
{\renewcommand\arraystretch{1.25}}
\begin{tabular} {ccccccc}
\toprule

Report                                 & 
\multicolumn{2}{c}{Fiscal Variable}    & 
\multicolumn{2}{c}{Source}             \\ 

\hline

\multirow{2}{*}{Full}                                          & 
\multicolumn{2}{p{6cm}}{\raggedright full or anytime or open, includes single and return, day single and day return.}                                                 & 
\multicolumn{2}{p{6cm}}{\raggedright Passengers can take any train.}    \\

\hline

\multirow{2}{*}{Reduced}                                                                                     &  
\multicolumn{2}{p{6cm}}{\raggedright reduced or off-peak or saver/super saver, includes single and return, day single and day return for off-peak and supper off-peak.}                                       & 
\multicolumn{2}{p{6cm}}{\raggedright Passengers can take any off-peak train. Peak time may vary from route to route.} \\

\hline

\multirow{2}{*}{Advance}                                                                     & 
\multicolumn{2}{p{6cm}}{\raggedright advance or apex, sold only as single tickets.} & 
\multicolumn{2}{p{6cm}}{\raggedright Passengers can only take the specific train of the ticket. Must be bought in advance and has limited availability}                                                                 \\

\bottomrule
\end{tabular}%
\caption*{Source: Own elaboration}
\end{table} 

